\documentclass{article}
\begin{document}
\begin{flushleft}
\emph{Problem}:

Square root convergents
Problem 57

It is possible to show that the square root of two can be expressed as an infinite continued fraction.
In the first one-thousand expansions, how many fractions contain a numerator with more digits than denominator?


\emph{Solution}:

The $\sqrt{2}$ can be expressed as 

\[ \sqrt{2} = 1 + \frac{1}{1 + \sqrt{2}} \approx  1 + \frac{1}{2} \] 

Call this expansion level 0.
This can be expanded again through recursion an arbitrary number of times

\[ \sqrt{2} = 1 + \frac{1}{2 + \frac{1}{\sqrt{2}}} \approx  1 + \frac{1}{2 + \frac{1}{2}} \]

Call this expansion level 1.
The point being that the greater the number of expansions, the closer the approximation of $\sqrt{2}$. 
Now let \emph{n} be our expansion level, let \emph{f(n)} be our function that returns the approximation
of $\sqrt{2}$ at a given level of expansion.

Now notice that at the deepest level of our continued fraction we are simply calculating 

\[\frac{1}{2+\frac{1}{2}}\]

Let's define another subroutine \emph{g(x)} 

\[g(x)=\frac{1}{2+x}\]

At the next level up $n-1,n>0$ we calculate

\[g \circ g(\frac{1}{2})\]

The insight I am trying to convey here is that for any exansion lebel \emph{n}, the value approximation can be expressed as
\[ f(n) = 1 + g^{\circ n}(\frac{1}{2}) \]


With this intuition, coding up a solution becomes trivial.

\end{flushleft}
\end{document}
